% setting document class according to SAGE guidelines
\documentclass[Royal,sageapa,times,doublespace]{sagej}

\usepackage{moreverb,url}
\usepackage[colorlinks,bookmarksopen,bookmarksnumbered,citecolor=red,urlcolor=red]{hyperref}

% mathematical packages
\usepackage{amsmath}
\usepackage{amssymb}

% table-related packages
\usepackage{multirow,booktabs,setspace,caption}
\usepackage{tikz}

%layout-related packages
\usepackage{indentfirst}

\newcommand\BibTeX{{\rmfamily B\kern-.05em \textsc{i\kern-.025em b}\kern-.08em
T\kern-.1667em\lower.7ex\hbox{E}\kern-.125emX}}

\def\volumeyear{2023}


\begin{document}

\runninghead{van Gerwen and Hessen}

\title{Assessing Fit of IRT models using a Randomisation LR Test and Fit Indices}

\author{Nina van Gerwen \affilnum{1} and Dave Hessen\affilnum{2}}

\affiliation{\affilnum{1}Utrecht University, NL \\
\affilnum{2}Utrecht University, NL}

\corrauth{Nina van Gerwen, Utrecht University,
Faculty of Social Sciences, Department of Methodology and Statistics,
Padualaan 14, Utrecht, 3584 CH, NL.}

\email{n.l.vangerwen@uu.nl}

\begin{abstract}
Abstract text.
\end{abstract}

\keywords{IRT, Goodness-of-fit test, fit indices}

\maketitle

\section{Introduction}
Item Response Theory (IRT) is an often used tool for measuring and analysing tests or questionnaires in psychological, educational and organisational practices due to the advantages it holds compared to Classical Test Theory (CTT). For example, compared to CTT, IRT has been shown to be better at individual change detection when using larger tests (Ruslan et al., 2016). Furthermore, IRT allows for more flexibility in analysing tests, where you can investigate all items separately instead of the test-level approach in CTT. The models used in IRT measure latent traits - constructs that are not directly observable (e.g., attitudes, intelligence, etc.) - through analysing answers to a test/questionnaire. The goal in IRT is to find the model that best describes the scores on the test items. To achieve this, the model must show a good fit. If a wrong model is used, it can lead to dire consequences such as faults in the validity of your measurement (Cri\c{s}an et al., 2017; Jiao \& Lau, 2003; Zhao \& Hambleton, 2017) and by extension your conclusion. Therefore, model-fit should be assessed after fitting an IRT model to the data gained from a questionnaire. \\
\indent There are mainly two procedures that can be applied to measure model-fit. First, model-fit can be assessed through goodness-of-fit tests. Commonly used goodness-of-fit tests are a $\chi^2$-difference test and Pearson's $\chi^2$-test. These tests, however, both suffer from issues. Pearson's test, for example, will not follow a $\chi^2$-distribution when many score pattern frequencies are missing or low, which is often the case, especially when test length increases. The $\chi^2$-difference test instead is difficult to use for the three-parameter (3PL) model. This is because generalisations of the 3PL model tend to suffer from consistency and estimation issues (Barton \& Lord, 1981; Loken \& Rulison, 2010). Another concern with the $\chi^2$-difference test is power, which can be viewed as both a blessing and a curse. This is because many goodness-of-fit tests suffer from a lack of power when sample size is low. However, when sample size increases, the power to reject any reasonable model that does not perfectly fit, also increases (Curran et al., 1996). \\
\indent Therefore, fit indices were introduced as a second procedure to assess model-fit. Fit indices are mathematical descriptives that indicate how well a model fits to the data. Note, however, that fit indices alone are not sufficient to infer whether a model fits well as they are not an inferential statistic and only describe the overall fit. Taken together with goodness-of-fit tests, fit indices do allow researchers to have a more comprehensive overview of model-fit. For example, when due to a lot of power, a goodness-of-fit test shows that the model should be rejected, fit indices can indicate whether the model is still reasonable. Previous research in IRT has seen the development of multiple fit indices. (e.g., $Y\text{-}Q_1$; Yen, 1981, $\text{RMSEA}_n$; Maydeu-Olivares \& Joe, 2014). For an overview of the use and limitations of some of the IRT fit indices, see Nye et al. (2020). Fit indices from Structural Equation Modeling can also be used in IRT, such as the SRMR. Although research has been done to examine the performance of some the Structural Equation Modeling fit indices in IRT (e.g., RMSEA \& SRMR; Maydeu-Olivares \& Joe, 2014), hardly any research has been done towards the use of the Tucker-Lewis Index (TLI; Tucker \& Lewis, 1973) and the Comparative Fit Index (CFI; Bentler, 1990) in an IRT context. Only one paper examines the CFI (Yang, 2020) and two papers investigate the TLI (Cai et al., 2021; Yang, 2020) in an IRT setting. The added value of the present study lies within the calculations of the fit indices. The calculations will be based on a complete-independence baseline model, which we argue is a more accurate baseline model. Furthermore, compared to Cai et al., we base the calculations on the $\chi^2$ information statistic and not a limited information statistic. Both of these factors can influence the performance and therefore should be investigated further. \\
\indent To summarise, in the field of IRT there are (a) not enough usable goodness-of-fit tests available to test the 3PL model and (b) scarce studies investigating the performance of the CFI and TLI, which we address.

\subsection{The present study}
In order to create a goodness-of-fit test to use for the 3PL in IRT and to better understand the possible uses of the CFI and TLI in an IRT setting, the present study answers the following three research questions through a simulation study:
\begin{enumerate}
\item{What sample size is necessary at different test lengths for the Likelihood Ratio (LR) Randomisation test to perform well?}
\item{How does the performance of the LR Randomisation test compare to the performance of a $\chi^2$-difference and Pearson's $\chi^2$-test?}
\item{What is the performance of the TLI and CFI with a complete-independence baseline model in IRT?}
\end{enumerate}
Results from the simulation study are reported to answer the research questions. Furthermore, empirical data from XX study are used to illustrate the value of the LR Randomisation test, the TLI and the CFI.

\section{Methods}
\subsection{Statistical background}
Before we share the methodology of the current study, let us first examine a brief summary on the statistical theory associated with our study. In IRT, the goal is to find a model that best describes scores on test items. To achieve this, IRT presupposes three assumptions: (1) conditional independence of items given the latent trait, denoted by $\theta$, (2) independence of observations and (3) the response to an item can be modeled by an item response function (IRF). An IRF is a mathematical equation that relates the probability to score a certain category on an item to $\theta$. In the present study, we consider IRT with unidimensional $\theta$, dichotomous test items and three different IRF. Below, you find the IRF for the three-parameter logistic model (3PL):

\begin{equation}
P(X_i = 1 | \theta, \alpha_{i}, \beta_{i}, \gamma_{i}) = \gamma_{i} + (1 - \gamma_{i}) \cdot 
\frac{e^{\alpha_{i}\theta - \beta_{i}}}{1 + e^{\alpha_{i}\theta - \beta_{i}}},
\end{equation}

where $X_i$ is a random variable indicating the response to item $i$. The probability of scoring a 1 on item $i$ in the 3PL model depends on (a) the latent variable, $\theta$, (b) the location parameter of the item $\beta_{i}$, which denotes how difficult the item is, (c) the scaling parameter of the item $\alpha_{i}$, which shows how well item $i$ discriminates between individuals who score a 0 and individuals who score a 1, and (d) an item-specific lower asymptote $\gamma_{i}$, which indicates whether there is a baseline probability of scoring a 1 (e.g., a multiple choice test with 4 options has a .25 baseline probability of scoring a 1). The two-parameter logistic model (2PL) is a special case of the 3PL, where $\gamma_{i}$ equals 0 for all items. The 2PL can then be specified even further to obtain the one-parameter logistic model (1PL), where it is assumed that $\alpha_{i}$ equals 1 for all items. The 1PL model is also commonly known as the Rasch model. \\
\indent Combining an IRF with the assumption of conditional independence allows us to model the probability of a complete score pattern to $k$ items simply by factoring the probabilities for each item:

\begin{equation}
P(\boldsymbol{X} = \boldsymbol{x} | \theta, \boldsymbol{\nu}) = \prod_{i=1}^{k} \{P(X_i = 1 | \theta, \boldsymbol{\nu})\}^{x_i} \cdot  \{1 - P(X_i = 1 | \theta, \boldsymbol{\nu}) \}^{1 - x_i},
\end{equation}

where $\boldsymbol{X}$ is now a random vector indicating a score pattern and $\boldsymbol{x}$ is the realisation of $\boldsymbol{X}$. Note that $\boldsymbol{\nu}$ is a vector containing item parameters for all $k$ items. We can take this even further by taking into account the assumption of independence of observations and the assumption that persons are randomly sampled from a population. The joint marginal probability of all score patterns in a given sample then becomes:

\begin{equation}
P(\boldsymbol{X} = \boldsymbol{x}) = \int \prod_{i=1}^{k} \{ P(\boldsymbol{X} = \boldsymbol{x} | \theta, \boldsymbol{\nu}) \} \,\phi(\theta)\,d\theta,
\end{equation}

where $\phi(\theta)$ is the univariate density of the latent variable $\theta$. In order to solve this equation, the density of $\theta$ has to be specified. In the current study, we assume $\theta$ to be a standard normal distribution. With the joint marginal probability, we can construct a likelihood function and estimate $\boldsymbol{\nu}$ through marginal maximum likelihood estimation:

\begin{equation}
\mathcal{L}(\boldsymbol{\nu}) = \prod_{\boldsymbol{x}} (P(\boldsymbol{X} = \boldsymbol{x}))^{n_{\boldsymbol{x}}}
\end{equation}

where $n_{\boldsymbol{x}}$ is the frequency of score pattern $\boldsymbol{x}$. Maximisation of $\mathcal{L}(\boldsymbol{\nu})$ would then lead to the estimation of the item parameters $\boldsymbol{\hat{\nu}}$. This is generally how an IRT model is fitted to data. In the current study, models are fitted using functions from the \textit{ltm} package in R (Rizopoulos, 2006), which approximates marginal maximum likelihood estimation through the Gauss-Hermite quadrature rule.
\subsection{Assessing Model Fit}
\indent After a model has been fitted as described above, the next step is usually to test how well the model fits the data. There are multiple options to assess model fit. Commonly, both goodness-of-fit tests and fit indices are used. The present study compares the performance of three goodness-of-fit tests: (1) a $\chi^2$-difference test, (2) Pearson's $\chi^2$-test and (3) our own developed test, a LR Randomisation test with the following formula:

\begin{equation}
\frac{max(\mathcal{L}_0)}{\prod_{j = 1}^g max(\mathcal{L}_j)},
\end{equation}

where $L_0$ is the likelihood of the chosen model for the whole dataset and $L_j$ is the likelihood of the chosen model for each group, gained by randomly assigning the observations to $g$ groups. According to Wilks' theorem (Wilks, 1938), this LR will then asymptotically follow a $\chi^2$-distribution. This allows the test to be used for Null Hypothesis Significance testing. The $\chi^2$-difference test is obtained in the following ways: the 1PL model is tested under the 2PL model, the 2PL model is tested under the 3PL model. For the 3PL model, the test is not used due to the limitations mentioned before. Finally, the Pearson's $\chi^2$-test is calculated through aggregating score patterns from the data and comparing the observed score pattern frequencies to the expected score pattern frequencies under the model. As for fit indices, we research the performance of the TLI and CFI in an IRT context. These models make use of a baseline model, our baseline model is a complete-independence model with the following IRF:

\begin{equation}
P(X_i = 1 | \beta_{i}) = \frac{e^{- \beta_{i}}}{1 + e^{- \beta_{i}}},
\end{equation}

where the probability of scoring a 1 on item $i$ is dependent only on the difficulty of the item and no longer on a latent variable. We argue that this is a correct baseline model, because the IRF entails that the joint probability distribution is simply the product of the marginal probability distributions and therefore the items will no longer correlate with one another (i.e., they are independent). Furthermore, the models also make use of a saturated model, where there are as many parameters as data points, leading to a perfect fit:

\begin{equation}
P(\boldsymbol{X} = \boldsymbol{x}) = \pi_{\boldsymbol{x}}.
\end{equation}

In the saturated model, there is no longer an IRF. Instead, perfect model fit is gained by allowing each score pattern to have their own parameter ($\pi_{\boldsymbol{X}}$), which is equal to the relative frequency of the score pattern. Using these two models, the fit indices can be calculated through the following formulae:

\begin{equation}
\text{CFI} = 1 - \frac{\chi^{2} - df}{\chi^{2}_{0} - df_0},
\end{equation}
\begin{equation}
\text{TLI} = 1 - \frac{\chi^{2}/df}{\chi^{2}_{0}/df_0},
\end{equation}

where in both equations, the numerator is a $\chi^2$-difference test between the tested model and the saturated model with $df$ degrees of freedom, and the denominator is a $\chi^2$-difference test between the tested model and the complete independence model with $df_0$ degrees of freedom.

\subsection{Data generation}
Data generation is done by first sampling person parameters, $\theta$, from a standard normal distribution. Then, a model is chosen as basis for the data generation. We choose to keep item parameters static over all simulations. For the difficulty parameter $\beta_i$, we choose the values -2, -1, 0, 1 and 2 for every repetition of five items. As for the discrimination parameter $\alpha_i$, we choose repetitions of the values 0.7, 0.85, 1, 1.15 and 1.3 per five items. Finally, for the pseudo-guessing parameter $\gamma_i$, we chose XXX. Then, probabilities are calculated for all items on a test, given $\theta$ and the chosen model's item parameters. Finally, a matrix of simulated responses to a test is created by sampling from a binomial distribution for every item, given each person. We replicated each simulation condition (see below) 500 times.

\subsection{Simulation design}
In order to answer the research questions, we conduct a simulation study that varies four factors: test length, sample size, model types and number of groups. For an overview of the conditions we used for the factors, see \textit{Table \ref{tab:1}}. This resulted in a total of 3 (test length) x 5 (sample size) x 3 (model type) = 45 conditions. \\
\indent In each replication of each condition, we obtain the results of the five goodness-of-fit tests (of which three are different versions of the LR Randomisation test) and the two fit indices. Performance of the different goodness-of-fit tests is then studied by estimating both type I error and power. Power is estimated when fitting and testing a different model to the data than the model used to generate the data. Type I error is estimated when fitting and testing the model that was used to generate the data. With these values, we compare the different type of tests with one another, where a test with lower power or higher type I error is noted as performing worse.
To measure the performance of the TLI and CFI, we calculate the proportion of times that the fit indices improve when the correct model is used compared to another model.

\begin{table}[htpb]
\caption{Overview of Simulation Conditions for Each Factor}
\begin{tabular}{ c c c }
\toprule
Factor & Conditions & Description \\
 \\
\midrule
\multicolumn{1}{l}{Test length} & 5 - 10 - 20 & \multicolumn{1}{l}{\shortstack{The total number of items that the test \\ will consist of}} \\ \\ 
\multicolumn{1}{l}{Sample size} & 20 - 50 - 100 - 200 - 500 & \multicolumn{1}{l}{\shortstack{The total number of observations that \\ will be available for each item}} \\ \\
\multicolumn{1}{l}{Model type} & 1PL - 2PL - 3PL & \multicolumn{1}{l}{\shortstack{The models that we will use as the basis for \\ both data generation and model-fitting}} \\ \\
\multicolumn{1}{l}{Number of groups} & 2 - 3 - 4 & \multicolumn{1}{l}{\shortstack{The number of groups that the total dataset \\ gets divided into for the \\ LR Randomisation test calculations}} \\

\bottomrule
\end{tabular}

\bigskip
\small\textit{Note}. 1PL = one-parameter logistic model; 2PL = two-parameter logistic model; 3PL = \\ three-parameter logistic model.
\label{tab:1}
\end{table}

\subsection{Empirical example}
To examine the possible use of our LR Randomisation test, the TLI and CFI, we estimated and interpreted their values on a real-life dataset, gained from XX, which contains information about XXX ($N = ...$). The dataset has XX items on a dichotomous scale that together measure XXX. We fitted the 3PL model with XX.

%\newpage

%\section{Results}

%\subsection{Empirical example}
%\section{Discussion}

%\nocite{*}
%\bibliographystyle{apalike}
%\bibliography{reportref}

\end{document}
